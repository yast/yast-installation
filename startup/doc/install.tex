\chapter{Possible installation methods}

\begin{itemize}
\item \textbf{\underline{Standard X11 based, UI:Qt or UI:Gtk}}\\
      Insert bootable CD and simply start the installation without
      any options set.
\item \textbf{\underline{Textmode with fbdev active, UI:ncurses}}\\
      Set the boot paramater \textbf{textmode=1}. In this case the
      YaST2 GUI will start in ncurses mode. The kernel framebuffer
      is active if supported.
\item \textbf{\underline{Textmode with vga active, UI:ncurses}}\\
      Select \textbf{F2 -> TextMode}. In this case the kernel framebuffer
      is switched off and the GUI will start in ncurses mode.
\item \textbf{\underline{Remote via VNC, UI:Qt}}\\
      Set the boot paramater \textbf{vnc=1 vncpassword=12345678}. Use
      the vncviewer on the remote side to connect to the VNC server.
      Additional information how to connect will be printed to the
      textconsole.
\item \textbf{\underline{Remote via SSH, UI:ncurses}}\\
      Set the boot parameter \textbf{usessh=1 sshpassword=12345}.
      After first reboot you need to login again and call
      YaST2.ssh which calls YaST2.call in continue mode to finish the
      installation.
\item \textbf{\underline{Serielle Konsole, UI:ncurses}}\\
      Set the boot parameter \textbf{console=ttyS0,115200} and plug in a
      serial cable between the serial interfaces of the two computers. Now
      call \textbf{screen /dev/ttyS0 115200} on the remote side and choose
      the appropriate terminal type from the list (normally type 7).
\item \textbf{\underline{Remote Display access, UI:Qt}}\\
      Set the boot parameter \textbf{Display\_IP=[IP or Name]} to an
      IP address or hostname whereas the corresponding machine has to accept
      X11 connections via port 6000
\end{itemize}
